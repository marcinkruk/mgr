\section{Methodology} \label{methodology}

\subsection{Granger causality} \label{granger-causality}
Framework for causality testing was first proposed by Granger \cite{granger69} in 1969.
The original formulation was based on a linear autoregressive model.
Consider two time series $X$ and $Y$ of length $n$, which are assumed to be weakly stationary.
Then, two linear predictors of $Y$ are constructed:

\begin{equation}
\bar{Y}_t = \sum_{i=0}^{p} \alpha_i Y_{t-i} + \varepsilon_{Y,t}
\end{equation}

\begin{equation}
\tilde{Y}_t = \sum_{i=0}^{p} a_i X_{t-i} \sum_{i=0}^{p} b_i Y_{t-i} + \varepsilon_{Y|X,t}
\end{equation}

where: $p$ is the order of AR model, $\alpha$, $a$, $b$ are linear coefficients and $\varepsilon_{Y,t}$ $\varepsilon_{Y|X,t}$ are residual errors of the models.

\subsubsection{Diks-Panchenko method}

\subsection{Wavelet decomposition}
\subsubsection{Daubechies wavelet}

\section{Dataset} \label{data}

\section{Results} \label{results}

\section{Conclusion} \label{conclusion}
