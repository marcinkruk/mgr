\newpage
\begin{center}
\textbf{Summary}
\end{center}

U.S. Dollar Index and crude oil price are important economic variables.
According to the current knowledge, their relation to one another is complex and time-varying.
Available research on this topic gives ambigous results.
Examining casusality between those can provide valuable insights to economists, policy makers and other market agents.

A multiscale linear and nonlinear Granger causality analysis was performed for U.S. Dollar exchange rate and crude oil price.
Time series were decomposed into multiple time scales using the discrete wavelet transform in Daubechies basis.
Linear Granger causality test and nonlinear Diks-Panchenko test were applied to each pair of bands of wavelet decomposed data.
Fourier phase randomisation method for surrogate data generation was employed to verify obtained results.

Some causality has been found in the analyzed time series, with no major patterns observed.
However, tests for surrogate data suggest relatively high false-positive rate.
Therefore, we refrain from stating definitive conclusions about causality between U.S. Dollar Index and oil price.
\vspace{2.5cm}
\begin{center}

\textbf{Keywords}\vspace{0.3cm}

Granger causality, time series analysis, wavelet decomposition, surrogate data, \mbox{currency exchange market}, crude oil market.\\
\vspace{2.5cm}

\textbf{Title of the thesis in Polish language}\vspace{0.3cm}

Wieloskalowa nieliniowa przyczynowość Grangera na rynkach walutowych
\end{center}

% TODO: Probably move this pagenumbering thing to the thesis.tex file.
\pagenumbering{gobble}
