\newpage
\begin{center}
\textbf{Streszczenie}
\end{center}

W pracy opisano metodę obliczania elementu macierzowego $\braket{\delta({\vec{r}_{12}})}$  w wybranej bazie funkcji próbnych dla molekuły H$_2$. Uzyskano wyniki dla szerokiego zakresu odległości międzyjądrowej $R$ i przeanalizowano przebieg tej zależności. Element ten wnosi wkład do poprawki do energii wynikającej z uwzględnienia efektów relatywistycznych.
\vspace{2.5cm}
\begin{center}
\textbf{Słowa kluczowe}\vspace{0.3cm}

Spektroskopia, molekuły dwuatomowe, metoda Ritza.\\
\vspace{2.5cm}
\textbf{Dziedzina pracy (kody według programu Socrates-Erasmus)}\vspace{0.3cm}

13.2 Fizyka\\
\vspace{2.5cm}
\textbf{Tytuł pracy w języku angielskim}\vspace{0.3cm}

Long-range asympotic behaviour of molecular hydrogen wavefunction
\end{center}

% TODO: Probably move this pagenumbering thing to the thesis.tex file.
\pagenumbering{gobble}
