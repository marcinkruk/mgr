\section{Motivation} \label{sec:motivation}

U.S. Dollar exchange rate and crude oil price are significant economic indexes.
Oil, literally and figuratively, fuels the contemporary economic system.
Because the U.S. is the World's biggest oil producer and a major exporter, their currency's exchange rate 
can have a significant impact on oil's demand. Moreover, the U.S. Dollar is the major invoicing currency
on international oil markets, making the connections between currency exchange rate and oil price even more
emphasized.

Quantifying causality between economic variables is crucial for better understanding of links between them.
This in turn allows economists to constuct better models, improve policies by govening bodies and may even 
provide competitive advantage for various market agents.

Significant number of research was conducted on relationships between U.S. Dollar and crude oil price.
Granger causality has been studied revealing mixed results. Relatively recent examples, containg comprehensive
overview are \cite{2010-oil-dollar, 2012-benhmad, 2017-oil-dollar, 2020-beckmann}.

The aim of this thesis is to deepen the understanding of causal links between U.S. Dollar exchange rate
and crude oil price by testing for both linear and nonlinear Granger causality.
Additional insights come from decomposing investigated time series into multiple time scales by wavelet
transform approach.
