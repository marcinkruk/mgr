\section{Datasets and software} \label{sec:data}

\subsection{Datasets}

Mothly data for crude oil price and U.S. dollar exchange rate for time period from 1974 to 2019 was analyzed.

Data for "U.S. Crude Oil First Purchase Price" comes from U.S. Energy Information Administration \cite{crude-oil-data}. The price is expressed in U.S. dollars per barrel.

Data referred to as "U.S. dollar index" is actually "Trade Weighted U.S. Dollar Index".
It's defined as "a weighted average of the foreign exchange value of the U.S. dollar against a subset of the broad index currencies that circulate widely outside the country of issue".
The index is set to 100 for March 1973.
The dataset was issued by The Federal Reserve Bank of St. Louis \cite{usd-data}.

Raw data plot is presented on \fref{fig:raw-data}

\begin{figure}[h]
	\includegraphics{../data/plot/raw}
	\caption{Plot of raw input data of crude oil price and U.S. Dollar Index.}
	\label{fig:raw-data}
\end{figure}


\subsection{Software}
Data analysis was performed using Python and C programs available on open source licenses.
NumPy \cite{numpy} and Pandas \cite{reback2020pandas} were used for array handling of input data.
Fourier transform algorithms from SciPy \cite{scipy} were used for surrogate data generation.
PyWavelet \cite{pywavelet} was used for wavelet decomposition of time series.
StatsModels \cite{statsmodels} has an implementation of linear Granger causality test.
C code from authors of \cite{diks-panchenko2004} was employed for non-linear causality test.
A script automating the whole process was developed using Python.
