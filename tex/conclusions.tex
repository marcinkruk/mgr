\section{Conclusions} \label{sec:conclusions}

Causality between U.S. Dollar index and oil price is a complex problem.
There are multiple identifiable factors suggesting influence on one another in several time scales.
On top of that, the external economial factors are plentiful and play a big role affecting those variables.
Studies show that connections between U.S. Dollar index and oil price can be time-varying and dependent on specific characteristics of the investigated economy \cite{2020-beckmann}. 

The methodology used in this study seems to be unable to distinguish those fine-grained contributing factors and establish high-fidelity results.
Some causality has been found in the analyzed data.
However, tests run for generated surrogate data suggest that many of those may be a false-positive.
Therefore, in the thesis, we refrain from stating definitive conclusions about causality between U.S. Dollar index and oil price.

This thesis is an example of using wavelet transform to analyze causality in multiple time scales, 
along with surrogate data to assess results fidelity.
