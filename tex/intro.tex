\section{Introduction}

Unraveling causal relations is a fundamental object of interest in science.
Causality testing is an important branch of time series analysis.
Granger \cite{granger69} stated a flexible theoretical problem definition for causality in his seminal paper in 1969.
This framework laid ground for abundance of causality test implementations among many fields:
physics \cite{inverse-ising,coupled-oscillators},
finance \cite{hiemstra-jones, gold-stock},
neuroscience \cite{Bullmore2009,causality-eeg,causality-visual},
environmental studies \cite{dogan2016co,ecological-economics}
and other \cite{social-media}.
The impact of this approach was later recognized awarding him the Nobel Prize in Economics in 2003 \cite{nobel2003}.

Spectral methods of time series analysis enable investigatation of its features in different frequency domains.
Wavelet approach allows to transform signals without loosing information from its original domain (usually spatial or temporal), overcoming limitations of Fourier analysis.
Wavelet analysis has proven to be a prominent tools in image compression \cite{jpeg2000}, 
pattern recognition \cite{pattern-recognition}, 
time series analysis \cite{multifractal-time-series} 
and other. % TODO: citations for examples

In this thesis a linear and non-linear Granger test is employed to investigate causal relations
in multiple time scales between U.S Dollar exchange rate and crude oil price.
\Fref{sec:motivation} contains literature review and motivation for research,
\fref{sec:methodology} decribes in detail methodology used for data analysis.
Data sets used in this thesis are characterized in \fref{sec:data}, along with software used for data analysis.
\Fref{sec:results} gathers important results and gives insights about discovered causal relationships.
Lastly, closing remarks are presented in \fref{sec:conclusions}.
