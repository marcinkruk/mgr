\documentclass[a4paper,12pt]{article}

\title{Wpływ oddziaływania van der Waalsa na strukturę energetyczną grafenu na podłożu metalicznym i izolującym}
\author{Marcin Kruk}
\date{}

\usepackage{amsmath}
\usepackage[T1]{fontenc} 
\usepackage[utf8]{inputenc}
\usepackage[polish]{babel}
\usepackage{graphicx}
\usepackage[numbers]{natbib}
\usepackage{indentfirst}
\usepackage{braket}
\usepackage{amsfonts}

\usepackage[stable]{footmisc}

%\usepackage{showframe}
%\usepackage[a4paper]{geometry}
\usepackage{fullpage}

\renewcommand\thesubsubsection{\thesubsection \alph{subsubsection}}
\numberwithin{equation}{section}
\numberwithin{table}{section}
%\renewcommand{\figurename}{Wykres}
\numberwithin{figure}{section}
\addto\captionspolish{\renewcommand{\figurename}{Wykres}}
\addto\captionspolish{\renewcommand{\tablename}{Tabela}}
%\DeclareMathAlphabet{\mathpzc}{OT1}{pzc}{m}{it}
\def\arraystretch{1.5}
\begin{document}

%
% STRONA TYTUŁOWA
%
\newpage
\thispagestyle{empty}
\vspace*{2cm}
\begin{center}
{\huge Uniwersytet Warszawski\\ \vspace{0.3cm}
Wydział Fizyki}\\
\vspace{1.5cm}
Marcin Kruk\\
Nr albumu: 320379\\
\vspace{1.5cm}
{\huge Wpływ oddziaływania van der Waalsa\\
	na strukturę energetyczną grafenu
	\mbox{na podłożu} metalicznym i izolującym}\\

\vspace{2cm}
Praca magisterska\\
na kierunku Fizyka
\end{center}
\vspace{3cm}
\hfill Praca wykonana pod kierunkiem \vspace{0.2cm}


\hfill prof. dra hab. Jacka Majewskiego

\hfill Instytut Fizyki Teoretycznej

\hfill Katedra Fizyki Materii Skondensowanej
\vfill
\begin{center}
{\large Warszawa, wrzesień 2016}\\
\end{center}
%
% pusta strona
%
%\newpage
%\thispagestyle{empty}
%\mbox{}
%
% OŚWIADCZENIA
%
\newpage
\thispagestyle{empty}
\vspace*{2cm}
\noindent
\emph{Oświadczenie kierującego pracą}
\vspace{0.6cm}


\noindent Oświadczam, że niniejsza praca została przygotowana pod moim kierunkiem i stwierdzam, że spełnia ona warunki do przedstawienia jej w postępowaniu o nadanie tytułu zawodowego.
\vspace{1cm}

\noindent ................... \hfill ........................................\\
 Data \hfill Podpis kierującego
\vspace{3cm}

\noindent \emph{Oświadczenia autora pracy}
\vspace{0.6cm}


\noindent Świadom odpowiedzalności prawnej oświadczam, że niniejsza praca dyplomowa została napisana przeze mnie i nie zawiera treści uzyskanych w sposób niezgodny z przepisami.
\vspace{0.3cm}

\noindent Oświadczam również, że przedstawiona praca nie była wcześniej przedmiotem procedur związanych z uzyskaniem tytułu zawodowego na wyższej uczelni.
\vspace{0.3cm}

\noindent Oświadczam ponadto, że niniejsza wersja pracy jest identyczna z załączoną wersją elektroniczną.
\vspace{1cm}

\noindent ................... \hfill ........................................\\
 Data \hfill Podpis autora pracy
\newpage
%
% pusta strona
%
%\newpage
%\thispagestyle{empty}
%\mbox{}
%
% STRESZCZENIE, SŁOWA KLUCZOWE
%
\newpage
\begin{center}
\textbf{Streszczenie}
\end{center}

W pracy opisano metodę obliczania elementu macierzowego $\braket{\delta({\vec{r}_{12}})}$  w wybranej bazie funkcji próbnych dla molekuły H$_2$. Uzyskano wyniki dla szerokiego zakresu odległości międzyjądrowej $R$ i przeanalizowano przebieg tej zależności. Element ten wnosi wkład do poprawki do energii wynikającej z uwzględnienia efektów relatywistycznych.
\vspace{2.5cm}
\begin{center}
\textbf{Słowa kluczowe}\vspace{0.3cm}

Spektroskopia, molekuły dwuatomowe, metoda Ritza.\\
\vspace{2.5cm}
\textbf{Dziedzina pracy (kody według programu Socrates-Erasmus)}\vspace{0.3cm}

13.2 Fizyka\\
\vspace{2.5cm}
\textbf{Tytuł pracy w języku angielskim}\vspace{0.3cm}

Long-range asympotic behaviour of molecular hydrogen wavefunction
\end{center}
\pagenumbering{gobble}
\newpage

%
% END OF DOCUMENT
%
\end{document}
